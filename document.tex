\documentclass[cn,hazy,sakura,screen,14pt]{elegantnote}

\graphicspath{{./figs/}}

\title{极简教程:基于git控制\LaTeX{}的\TeX{}版本管理}


\author{Z. Fang}
\institute{Chem. FDU}

\version{1.00}
\date{\zhtoday}


%\definecolor{geyecolor}{RGB}{199,237,204}
%\pagecolor{geyecolor}

\begin{document}
	
	\maketitle
	
	\centerline{
		\includegraphics[width=0.2\textwidth]{cover_image.jpg}
	}
	
	
	\section{Preface}
	
	本文的主要内容为如何基于git对\TeX{}进行版本控制。众所周知,u1s1\footnote{“有一说一”}, \LaTeX{}十分好用,你甚至可以2个小时内完全掌握\footnote{大雾。指某本教程。}。git呢,就是github的那个git,使用之前需要下载一个\footnote{https://git-scm.com/}(这不是废话吗)。行吧,不想听废话直接跳过这一部分好了。
	
	如果只是怕写论文而不想建那么多个“最终版”的话,仅仅需要熟记以下几个命令:
	\begin{itemize}
		\item \$ git log
		\item \$ git add
		\item \$ git commit
		\item \$ git tag
		\item \$ git reset
	\end{itemize}
	
	
	\section{第0步,写你的\TeX{}}
	这个就无所谓你用什么写啦,什么TexStudio,TexWork,甚至txt写好再编译都一样的,快乐就行。

	\section{第一步,新建git repository}
	在下载git后,右键空白处,“git bash here”,就可以进入命令行模式。
	新建的话直接输入
	\begin{lstlisting}
		\$ git init %初始化git respiratory,i.e.,将当前目录定义为一个repository
	\end{lstlisting}	
	
	\begin{remark}
	最好建一个文件夹,将写\TeX{}要用的所有资源放进去,然后直接在里面Bash就行啦。
	\end{remark}
	
	
	\section{第二步,将文件加入暂存区}
	将文件加入暂存区\footnote{不考虑Remote的话,git分workspace(你的\TeX{}),index和repository}的命令为:
	\begin{lstlisting}
		\$ git add .  %提交目录下的所有文件至暂存区
		\$ git add 你的文件名.文件类型  %提交目录下的某文件至暂存区,e.g.:git add Hello.docx	
	\end{lstlisting}

	\section{第三步,暂存区文件提交至repository}
	这一步需要的命令为:
	\begin{lstlisting}
		\$ git commit -m "版本更新日志"	
	\end{lstlisting}	
	commit之后就vans\footnote{即“万事”,玩谐音梗,我扣钱。}大吉了!

	\begin{note}
		你还可以给你的文档加tag:
		\begin{lstlisting}
		\$ git tag  %显示所有tag
		\$ git tag XXX 	%在当下的commit建一个tag
		\$ git tag XXX XXX 	%给指定commit建tag
		\end{lstlisting}
	\end{note}
	
	\section{回撤版本}	
	写文章总会遇到不如意嘛,比如上个版本写的比较好......这个时候,你就需要回退版本了。使用
	\begin{lstlisting}
		git reset --hard HEAD^ 	回退一个版本(将版本退回上一个commit的状态):
		git reset --hard HEAD^^ 	回退两个版本:
		git reset --hard HEAD~100  	回退100个版本:
		git reset --hard 版本号  	回退到指定版本:	
	\end{lstlisting}
	版本号可以通过git log查询。git log就是你git的所有记录,你可以慢慢看这里面的日志。
	
	\section{致谢}
	谢谢你!
		
	
	\section{常见问题 FAQ}
	\begin{enumerate}[label=\arabic*).]
		\item \textit{有什么问题?}\\
		你没有问题。
	\end{enumerate}
	
	\section{示例}
	略。或许此处应该有个示例吧。
	
\end{document}
% Local Variables:
% TeX-engine: xetex
% TeX-command-extra-options: "-interaction=nonstopmode"
% End:
